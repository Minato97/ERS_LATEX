\begin{longtable}{|p{0.28\textwidth}|p{0.67\textwidth}|}
\caption{Desacoplamiento de la Rúbrica de Evaluación} \label{tab:rnf-1} \\
\hline
\endfirsthead
\multicolumn{2}{c}%
{\tablename\ \thetable\ -- \textit{Continuación}} \\
\hline
\endhead
\hline
\multicolumn{2}{r}{\textit{Continúa en la siguiente página}} \\
\endfoot
\hline
\endlastfoot
\textbf{Id del requerimiento:} & RNF-1 \\
\hline
\textbf{Nombre:} & Desacoplamiento de la Rúbrica de Evaluación \\
\hline
\textbf{Categoría:} & Seguridad \\
\hline
\textbf{Descripción:} & El sistema no debe contener la lógica matemática ni los criterios específicos de la rúbrica de evaluación (preguntas específicas y puntajes) en su código. Debe permitir ingresar la calificación final y subir el archivo de la rúbrica llena como evidencia. Esto es para evitar modificar el sistema si el comité decide cambiar la forma de evaluar. \\
\hline
\textbf{Pre-condiciones:} & El usuario debe estar autenticado. \newline El usuario debe tener un rol asignado. \\
\hline
\textbf{Post-condiciones:} & El usuario solo puede acceder a las funciones autorizadas según su rol. \\
\hline
\textbf{Criterios de aceptación:} & Se valida que un usuario sin permisos no pueda acceder a módulos restringidos. \newline Se bloquea el acceso ante intentos no autorizados. \newline Las sesiones se gestionan mediante JWT vigente. \\
\hline
\textbf{Prioridad:} & Alta \\
\hline
\textbf{Estabilidad:} & Alta \\
\hline
\end{longtable}

\vspace{0.5cm}
\begin{longtable}{|p{0.28\textwidth}|p{0.67\textwidth}|}
\caption{Independencia de Plazos (Sin Cronómetros)} \label{tab:rnf-2} \\
\hline
\endfirsthead
\multicolumn{2}{c}%
{\tablename\ \thetable\ -- \textit{Continuación}} \\
\hline
\endhead
\hline
\multicolumn{2}{r}{\textit{Continúa en la siguiente página}} \\
\endfoot
\hline
\endlastfoot
\textbf{Id del requerimiento:} & RNF-2 \\
\hline
\textbf{Nombre:} & Independencia de Plazos (Sin Cronómetros) \\
\hline
\textbf{Categoría:} & Rendimiento \\
\hline
\textbf{Descripción:} & El sistema no debe cerrar etapas o procesos automáticamente basándose en fechas preestablecidas (hard-coded). El flujo debe avanzar por acción humana (ej. el comité guarda la calificación), permitiendo que cada proyecto avance a su propio ritmo. \\
\hline
\textbf{Pre-condiciones:} & El sistema debe estar funcionando correctamente. \\
\hline
\textbf{Post-condiciones:} & El usuario puede completar tareas principales sin requerir asistencia externa. \\
\hline
\textbf{Criterios de aceptación:} & Los usuarios pueden navegar entre módulos sin perderse. \newline Los textos, botones e iconos tienen significado claro. \newline Los procesos principales se completan en menos de 3 pasos. \\
\hline
\textbf{Prioridad:} & Alta \\
\hline
\textbf{Estabilidad:} & Alta \\
\hline
\end{longtable}

\vspace{0.5cm}
\begin{longtable}{|p{0.28\textwidth}|p{0.67\textwidth}|}
\caption{Gestión Genérica de Convocatorias} \label{tab:rnf-3} \\
\hline
\endfirsthead
\multicolumn{2}{c}%
{\tablename\ \thetable\ -- \textit{Continuación}} \\
\hline
\endhead
\hline
\multicolumn{2}{r}{\textit{Continúa en la siguiente página}} \\
\endfoot
\hline
\endlastfoot
\textbf{Id del requerimiento:} & RNF-3 \\
\hline
\textbf{Nombre:} & Gestión Genérica de Convocatorias \\
\hline
\textbf{Categoría:} & Portabilidad \\
\hline
\textbf{Descripción:} & El sistema debe ser capaz de manejar múltiples convocatorias (ej. anuales) o permitir el registro de proyectos sin una convocatoria activa (registro abierto/permanente). \\
\hline
\textbf{Pre-condiciones:} & El servidor debe tener acceso a la base de datos. \newline El sistema debe operar bajo carga normal. \\
\hline
\textbf{Post-condiciones:} & Las respuestas son entregadas dentro del tiempo aceptable. \\
\hline
\textbf{Criterios de aceptación:} & 80 por ciento de las peticiones responden en un tiempo igual o menor a 5 segundos. \newline Soporta N usuarios concurrentes sin caída del rendimiento. \\
\hline
\textbf{Prioridad:} & Alta \\
\hline
\textbf{Estabilidad:} & Media \\
\hline
\end{longtable}

\vspace{0.5cm}
\begin{longtable}{|p{0.28\textwidth}|p{0.67\textwidth}|}
\caption{Visibilidad del Estado del Proyecto} \label{tab:rnf-4} \\
\hline
\endfirsthead
\multicolumn{2}{c}%
{\tablename\ \thetable\ -- \textit{Continuación}} \\
\hline
\endhead
\hline
\multicolumn{2}{r}{\textit{Continúa en la siguiente página}} \\
\endfoot
\hline
\endlastfoot
\textbf{Id del requerimiento:} & RNF-4 \\
\hline
\textbf{Nombre:} & Visibilidad del Estado del Proyecto \\
\hline
\textbf{Categoría:} & Usabilidad \\
\hline
\textbf{Descripción:} & El usuario (investigador) debe poder visualizar claramente en qué etapa del proceso se encuentra su propuesta (ej. "En revisión", "Comité de Ética", "Dictaminado"), similar a un rastreo de paquetería o compras en línea. \\
\hline
\textbf{Pre-condiciones:} & Los módulos actuales deben estar correctamente integrados. \\
\hline
\textbf{Post-condiciones:} & Nuevos módulos o servicios pueden añadirse sin reestructurar el sistema. \\
\hline
\textbf{Criterios de aceptación:} & Integración de nuevos módulos con mínimo esfuerzo. \newline No requiere modificaciones profundas para ampliarse. \newline La base de datos soporta incremento progresivo de usuarios \\
\hline
\textbf{Prioridad:} & Media \\
\hline
\textbf{Estabilidad:} & Media \\
\hline
\end{longtable}

\vspace{0.5cm}
\begin{longtable}{|p{0.28\textwidth}|p{0.67\textwidth}|}
\caption{Prevención de Errores en Captura (Catálogos)} \label{tab:rnf-5} \\
\hline
\endfirsthead
\multicolumn{2}{c}%
{\tablename\ \thetable\ -- \textit{Continuación}} \\
\hline
\endhead
\hline
\multicolumn{2}{r}{\textit{Continúa en la siguiente página}} \\
\endfoot
\hline
\endlastfoot
\textbf{Id del requerimiento:} & RNF-5 \\
\hline
\textbf{Nombre:} & Prevención de Errores en Captura (Catálogos) \\
\hline
\textbf{Categoría:} & Confiabilidad \\
\hline
\textbf{Descripción:} & Para campos críticos como "Productos Entregables" (artículos, libros, tesis), el sistema debe usar listas de selección (select boxes/menús desplegables) predefinidas en lugar de campos de texto libre, para evitar errores tipográficos y facilitar la generación de estadísticas. \\
\hline
\textbf{Pre-condiciones:} & El navegador debe estar actualizado. \\
\hline
\textbf{Post-condiciones:} & La interfaz y funciones se muestran adecuadamente en todos los navegadores soportados. \\
\hline
\textbf{Criterios de aceptación:} & Pruebas de compatibilidad superadas en los tres navegadores. \newline No existen diferencias significativas en el diseño ni errores de ejecución. \\
\hline
\textbf{Prioridad:} & Media \\
\hline
\textbf{Estabilidad:} & Media \\
\hline
\end{longtable}

\vspace{0.5cm}
\begin{longtable}{|p{0.28\textwidth}|p{0.67\textwidth}|}
\caption{Capacidad de Guardado Borrador} \label{tab:rnf-6} \\
\hline
\endfirsthead
\multicolumn{2}{c}%
{\tablename\ \thetable\ -- \textit{Continuación}} \\
\hline
\endhead
\hline
\multicolumn{2}{r}{\textit{Continúa en la siguiente página}} \\
\endfoot
\hline
\endlastfoot
\textbf{Id del requerimiento:} & RNF-6 \\
\hline
\textbf{Nombre:} & Capacidad de Guardado Borrador \\
\hline
\textbf{Categoría:} & Mantenibilidad \\
\hline
\textbf{Descripción:} & El sistema debe permitir a los usuarios (evaluadores o investigadores) guardar su progreso parcialmente antes de enviar la información final. Debe existir una confirmación explícita antes del envío definitivo, tras lo cual no se permite edición. \\
\hline
\textbf{Pre-condiciones:} & Debe existir un diseño UI definido o un manual de identidad visual. \\
\hline
\textbf{Post-condiciones:} & Toda la interfaz usa la paleta establecida de forma consistente. \\
\hline
\textbf{Criterios de aceptación:} & No se utilizan colores fuera de la paleta definida. \newline La paleta es accesible y cumple contraste mínimo WCAG AA. \newline NOTA: La WCAG (Web Content Accessibility Guidelines) son normas internacionales que indican cómo hacer accesible un sitio web para personas con discapacidad visual o con baja visión. El nivel AA es el nivel estándar recomendado para la mayoría de los sistemas y aplicaciones web. \\
\hline
\textbf{Prioridad:} & Baja \\
\hline
\textbf{Estabilidad:} & Media \\
\hline
\end{longtable}

\vspace{0.5cm}
\begin{longtable}{|p{0.28\textwidth}|p{0.67\textwidth}|}
\caption{Control de Acceso Basado en Roles} \label{tab:rnf-7} \\
\hline
\endfirsthead
\multicolumn{2}{c}%
{\tablename\ \thetable\ -- \textit{Continuación}} \\
\hline
\endhead
\hline
\multicolumn{2}{r}{\textit{Continúa en la siguiente página}} \\
\endfoot
\hline
\endlastfoot
\textbf{Id del requerimiento:} & RNF-7 \\
\hline
\textbf{Nombre:} & Control de Acceso Basado en Roles \\
\hline
\textbf{Categoría:} & Escalabilidad \\
\hline
\textbf{Descripción:} & El Responsable del Proyecto solo debe tener acceso a ver y gestionar sus propios proyectos. El Comité Científico debe tener permisos para cargar evaluaciones, pero se debe restringir la edición de calificaciones propias si un miembro del comité también es participante (evitar conflicto de interés). \\
\hline
\textbf{Pre-condiciones:} & La API debe estar activa y accesible. \newline Las rutas deben estar correctamente documentadas. \\
\hline
\textbf{Post-condiciones:} & Permite intercambio de datos con sistemas externos sin errores. \\
\hline
\textbf{Criterios de aceptación:} & La API responde en formato JSON válido. \newline Los sistemas externos pueden consultar o enviar datos correctamente. \\
\hline
\textbf{Prioridad:} & Alta \\
\hline
\textbf{Estabilidad:} & Media \\
\hline
\end{longtable}

\vspace{0.5cm}
\begin{longtable}{|p{0.28\textwidth}|p{0.67\textwidth}|}
\caption{Integridad de los Datos} \label{tab:rnf-8} \\
\hline
\endfirsthead
\multicolumn{2}{c}%
{\tablename\ \thetable\ -- \textit{Continuación}} \\
\hline
\endhead
\hline
\multicolumn{2}{r}{\textit{Continúa en la siguiente página}} \\
\endfoot
\hline
\endlastfoot
\textbf{Id del requerimiento:} & RNF-8 \\
\hline
\textbf{Nombre:} & Integridad de los Datos \\
\hline
\textbf{Categoría:} & Mantenibilidad \\
\hline
\textbf{Descripción:} & Una vez que un revisor o investigador hace clic en "Enviar" o "Aceptar" final, el registro debe bloquearse para edición para asegurar que la evaluación no sea alterada posteriormente sin un proceso de desbloqueo administrativo. \\
\hline
\textbf{Pre-condiciones:} & El usuario accede desde un dispositivo con navegador web. \\
\hline
\textbf{Post-condiciones:} & La interfaz se ajusta según el tamaño de pantalla sin afectar la usabilidad. \\
\hline
\textbf{Criterios de aceptación:} & No aparecen desplazamientos horizontales innecesarios. \newline Los elementos se reordenan correctamente en vista móvil. \newline El sistema mantiene su funcionalidad completa en cualquier dispositivo. \\
\hline
\textbf{Prioridad:} & Alta \\
\hline
\textbf{Estabilidad:} & Media \\
\hline
\end{longtable}

\vspace{0.5cm}
\begin{longtable}{|p{0.28\textwidth}|p{0.67\textwidth}|}
\caption{Validación de Usuarios Únicos} \label{tab:rnf-09} \\
\hline
\endfirsthead
\multicolumn{2}{c}%
{\tablename\ \thetable\ -- \textit{Continuación}} \\
\hline
\endhead
\hline
\multicolumn{2}{r}{\textit{Continúa en la siguiente página}} \\
\endfoot
\hline
\endlastfoot
\textbf{Id del requerimiento:} & RNF-09 \\
\hline
\textbf{Nombre:} & Validación de Usuarios Únicos \\
\hline
\textbf{Categoría:} & Rendimiento \\
\hline
\textbf{Descripción:} & El sistema debe validar que los profesores ya capturados en la base de datos no se dupliquen al ser registrados en nuevos proyectos. \\
\hline
\textbf{Pre-condiciones:} & El usuario (Investigador o Administrador) intenta agregar un integrante a un proyecto. \newline Existe una base de datos de usuarios previamente registrados. \\
\hline
\textbf{Post-condiciones:} & El sistema asocia el registro existente del profesor al nuevo proyecto sin crear un nuevo ID de usuario. \newline Se mantiene la integridad de la base de datos (una sola "persona" lógica por registro) \\
\hline
\textbf{Criterios de aceptación:} & Al ingresar un identificador único (Correo Electrónico, Número de Empleado o CVU), el sistema debe buscar coincidencias antes de guardar. \newline Si el identificador ya existe, el sistema debe mostrar el nombre del usuario encontrado y permitir seleccionarlo. \newline El sistema debe bloquear la creación de un nuevo registro si el identificador clave ya existe en la base de datos. \\
\hline
\textbf{Prioridad:} & Alta \\
\hline
\textbf{Estabilidad:} & Alta \\
\hline
\end{longtable}

\vspace{0.5cm}
\begin{longtable}{|p{0.28\textwidth}|p{0.67\textwidth}|}
\caption{Plataforma Web} \label{tab:rnf-10} \\
\hline
\endfirsthead
\multicolumn{2}{c}%
{\tablename\ \thetable\ -- \textit{Continuación}} \\
\hline
\endhead
\hline
\multicolumn{2}{r}{\textit{Continúa en la siguiente página}} \\
\endfoot
\hline
\endlastfoot
\textbf{Id del requerimiento:} & RNF-10 \\
\hline
\textbf{Nombre:} & Plataforma Web \\
\hline
\textbf{Categoría:} & Usabilidad \\
\hline
\textbf{Descripción:} & El sistema debe ser accesible a través de un navegador web. \\
\hline
\textbf{Pre-condiciones:} & El usuario cuenta con un dispositivo con conexión a internet (PC, Laptop, Tablet). \newline El usuario tiene instalado un navegador web estándar. \\
\hline
\textbf{Post-condiciones:} & La interfaz gráfica se carga correctamente y es funcional sin necesidad de instalar software adicional en el cliente. \\
\hline
\textbf{Criterios de aceptación:} & El sistema debe renderizar correctamente en las versiones estables más recientes de Google Chrome, Mozilla Firefox, Microsoft Edge y Safari. \newline No se deben requerir plugins externos (como Flash o Java Applets) para su funcionamiento. \newline El diseño debe ser responsivo (ajustarse legiblemente) a resoluciones de pantalla estándar (mínimo 1366x768). \\
\hline
\textbf{Prioridad:} & Alta \\
\hline
\textbf{Estabilidad:} & Alta \\
\hline
\end{longtable}

\vspace{0.5cm}
\begin{longtable}{|p{0.28\textwidth}|p{0.67\textwidth}|}
\caption{Gestión de Archivos} \label{tab:rnf-11} \\
\hline
\endfirsthead
\multicolumn{2}{c}%
{\tablename\ \thetable\ -- \textit{Continuación}} \\
\hline
\endhead
\hline
\multicolumn{2}{r}{\textit{Continúa en la siguiente página}} \\
\endfoot
\hline
\endlastfoot
\textbf{Id del requerimiento:} & RNF-11 \\
\hline
\textbf{Nombre:} & Gestión de Archivos \\
\hline
\textbf{Categoría:} & Portabilidad \\
\hline
\textbf{Descripción:} & El sistema debe soportar la carga y almacenamiento de archivos en formato PDF, tanto para las propuestas iniciales como para las evidencias de productos (artículos, patentes). \\
\hline
\textbf{Pre-condiciones:} & El usuario se encuentra en una pantalla de carga (Registro de Propuesta, Correcciones o Carga de Evidencias). \newline El archivo a subir existe en el dispositivo local del usuario. \\
\hline
\textbf{Post-condiciones:} & El archivo queda almacenado en el servidor (o servicio de almacenamiento) y vinculado indisolublemente al registro del proyecto/producto correspondiente. \\
\hline
\textbf{Criterios de aceptación:} & El sistema debe permitir únicamente archivos con extensión .pdf. \newline Si el usuario intenta subir otro formato (.doc, .jpg), el sistema debe mostrar un mensaje de error y rechazar la carga. \newline El sistema debe validar un tamaño máximo de archivo (ej. 10 MB o el límite que defina infraestructura) para prevenir saturación. \newline El archivo subido debe poder ser visualizado o descargado posteriormente por los roles autorizados. \\
\hline
\textbf{Prioridad:} & Alta \\
\hline
\textbf{Estabilidad:} & Media \\
\hline
\end{longtable}

\vspace{0.5cm}
\begin{longtable}{|p{0.28\textwidth}|p{0.67\textwidth}|}
\caption{Escalabilidad de Productos} \label{tab:rnf-12} \\
\hline
\endfirsthead
\multicolumn{2}{c}%
{\tablename\ \thetable\ -- \textit{Continuación}} \\
\hline
\endhead
\hline
\multicolumn{2}{r}{\textit{Continúa en la siguiente página}} \\
\endfoot
\hline
\endlastfoot
\textbf{Id del requerimiento:} & RNF-12 \\
\hline
\textbf{Nombre:} & Escalabilidad de Productos \\
\hline
\textbf{Categoría:} & Escalabilidad \\
\hline
\textbf{Descripción:} & El sistema debe permitir la opción "Otro" en los catálogos de productos con un campo de especificación, previendo que surjan nuevos tipos de entregables no contemplados inicialmente. \\
\hline
\textbf{Pre-condiciones:} & El Investigador está en la sección de "Registro de Metas/Compromisos". \newline El producto que desea registrar no se encuentra en el listado estándar. \\
\hline
\textbf{Post-condiciones:} & Se registra un compromiso con una categoría genérica ("Otro") pero con una descripción específica guardada en la base de datos. \\
\hline
\textbf{Criterios de aceptación:} & El menú desplegable (select box) de productos debe incluir la opción "Otro" al final de la lista. \newline Al seleccionar "Otro", debe aparecer obligatoriamente un campo de texto libre habilitado para capturar el nombre del producto (ej. "Prototipo Industrial"). \newline El sistema no debe permitir guardar si se selecciona "Otro" y el campo de especificación está vacío. \newline En los reportes, este ítem debe mostrarse con el texto capturado por el usuario, no solo como "Otro". \\
\hline
\textbf{Prioridad:} & Media \\
\hline
\textbf{Estabilidad:} & Media \\
\hline
\end{longtable}

\vspace{0.5cm}
\begin{longtable}{|p{0.28\textwidth}|p{0.67\textwidth}|}
\caption{Historial de Versiones/Iteraciones} \label{tab:rnf-13} \\
\hline
\endfirsthead
\multicolumn{2}{c}%
{\tablename\ \thetable\ -- \textit{Continuación}} \\
\hline
\endhead
\hline
\multicolumn{2}{r}{\textit{Continúa en la siguiente página}} \\
\endfoot
\hline
\endlastfoot
\textbf{Id del requerimiento:} & RNF-13 \\
\hline
\textbf{Nombre:} & Historial de Versiones/Iteraciones \\
\hline
\textbf{Categoría:} & Rendimiento \\
\hline
\textbf{Descripción:} & El sistema debe permitir y mantener el historial de reenvíos (versiones corregidas) si el proyecto fue aprobado con condiciones o requiere mejoras, sin sobrescribir destructivamente la trazabilidad del proceso anterior. \\
\hline
\textbf{Pre-condiciones:} & Un proyecto tiene estatus "Aprobado con Observaciones" (o requiere corrección). \newline El investigador sube un nuevo archivo PDF corregido. \\
\hline
\textbf{Post-condiciones:} & El nuevo archivo se marca como la versión "Actual/Vigente". \newline El archivo anterior cambia su estado a "Histórico/Obsoleto" pero permanece accesible para auditoría. \\
\hline
\textbf{Criterios de aceptación:} & El sistema debe almacenar ambos archivos (el original y el corregido) en la base de datos/file system. \newline El sistema debe registrar la fecha y hora de la nueva subida. \newline El Administrador debe poder ver un historial que liste todas las versiones subidas de la propuesta. \newline Para efectos de evaluación actual, el sistema debe presentar por defecto la última versión subida. \\
\hline
\textbf{Prioridad:} & Alta \\
\hline
\textbf{Estabilidad:} & Media \\
\hline
\end{longtable}

\vspace{0.5cm}
